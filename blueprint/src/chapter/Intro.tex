\chapter{Introduction}

Jacquet-Langlands suggests that certain classical modular forms should be related to modular forms
on quaternion algebras. This is because classical modular forms can be viewed as automorphic forms
for $M_2 (\Q)$; which is also a quaternion algebra. Compared to the analytic defintion of classical
modular forms, for definite quaternion algebras, the definition of quaternionic modular forms is
much nicer to work with - specifically, being of a concrete combinatorial nature. Thus, to study
(some) classical modular forms, it may be easier to study quaternionic modular forms.

In this project the aim is to define such quaternionic modular forms over definite quaternion
algebras and then prove some results surrounding them in Lean. Specifcally, we aim to formalise
Daniel Jacobs' thesis "slopes of compact Hecke operators", which says that $p$-adic valuations of the
eigenvalues of certain Hecke operators over quaternionic modular forms are in arithmetic
progression.

\subsection{What is being formalised?}

In attempting to formalise Daniel Jacobs' theis paper we need to first build up foundation material
before we can start working on his paper. This includes:

\begin{enumerate}
  \item restricted power series;
  \item the Gauss norm; and
  \item Newton polygons.
\end{enumerate}

Each of these can be found in its own chapter of the blueprint. Where we introduce the topic and
state what we need in Lean. This is by no means the entirity of the field, but should be sufficient
for what we need.

After what we are dubbing the foundational material, we need to start working on results on
quaternionic modular forms. This includes:
\begin{enumerate}
  \item their definition;
  \item a result on their finite dimensionality (Fujisaki's lemma); and
  \item their Hecke operators.
\end{enumerate}

Finally, with this we can hopefully start formalising the results from Jacobs' thesis.
