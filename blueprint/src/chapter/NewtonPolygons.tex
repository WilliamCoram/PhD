\section{Newton Polygons}

In this section we will determine a way to check for zeroes of a power series via constructing
something called a Newton polygon. We will follow \cite[Section~7.4]{Gouvea} closely.

\subsection{Polynomials}

Suppose $p$ is a prime number and consider $K = \Q_p$. We can generalise everything here to
discretely valued fields, but for notation it is enough to consider $\Q_p$.

Consider a polynomial $f$ over $K$, we can define the Newton polygon of $f$ in the following way.

\begin{definition}
    The \emph{Newton polygon} of $f = \sum_{i=0}^n a_i x^i$ is the lower boundary of the convex hull
    of the set of points $(i,\nu(a_i))$.
    Here by convex hull we mean the smallest convex polygon that contains all points.
\end{definition}

This is easily constructable by the following algoritheorem, \cite[Pages~210-211]{Gouvea} .

\begin{enumerate}
    \item Plot the points $\{(i,\nu (i) )\}$.
    \item Start with the vertical half-line down from the point with the smallest $i$.
    \item Rotate that line counter-clockwise until it hits one of the points we have plotted.
    \item Break the line at that point and continue rotation the remaining part until another point
    is hit.
    \item Continue until all the points have been hit or lie strictly above a portion of the
    polygon.
\end{enumerate}

As mentioned, the study of Newton polygons is deeply connected to studying the zeroes of $f(x)$,
thus we will assume $f(0) \neq 0$. That is, we factor out any powers of $x$ dividing $f$, which
means our initial vertical line starts on the $y$-axis.

Moreover, we will also assume $a_0 = 1$. This is because if we factor
$f(x) = \sum_{i = 0}^n a_i x^i$ as $f(x) = a_0 \cdot g(x)$, where
$ g(x) = 1 + \sum_{i=1}^n \frac{a_i}{a_0}x^i$, it is not hard to see that the resulting Newton
polygon of $g(x)$ is the Newton polygon of $f(x)$ shifted down by $\nu (a_0)$.
This follows from observing $\nu (\frac{a_i}{a_0}) = \nu(a_i) - \nu (a_0)$.

Now we have defined what Newton polygons are, we want to explain why we care about them and their
slopes, and why they are connected with studying the zeros of the polynomial.

We give the following definitions which will be relevant.
\begin{definition}
    \begin{enumerate}
        \item The \emph{slopes} of the line segments appearing in the Newton polygon are called the
        slopes of $f(x)$.
        \item The \emph{length} of a slope is the length of the projection of the line segment with
        that slope onto the $x$-axis.
        \item The \emph{breaks} are the values of $i$ which are vertices of the Newton polygon.
    \end{enumerate}
\end{definition}

\begin{theorem}
  The slopes of a Newton polgyon form an increasing sequence.
\end{theorem}

\begin{proof}
    Consider the first point, $(0,0)$ and let the first break occur at $(i,m i)$, so
    that the first slope is $m$. Then by construction no points lie below this line; that is
    $(j, \nu(j))$ lies above the line for all $j$. In other words, this means $\nu (a_j) \geq m j$
    for every j. But, if the second slope was smaller we would have to have $(j, \nu(a_j))$ lying
    below this line, which contradicts above.
\end{proof}

The connection to roots is then witnessed in the following theorem, \cite[Theorem~7.4.7]{Gouvea}.
\begin{theorem}
    \label{NPPoly}
    Let $f(x) = 1 + a_1 x + \dots + a_nx^n$ be a polynomial over $K$, and let $m_1, \dots, m_r$ be
    the slopes of its Newton polygon (in increasing order). Let $i_1, \dots, i_r$ be the
    corresponding lengths. Then for each $k$, $1 \leq k \leq r$, $f(x)$ has exactly $i_k$ roots of
    valuation $m_k$ (counting multiplicity).
\end{theorem}

\begin{proof}
  Need to write up.
\end{proof}

Therefore, by computing the Newton polygon of a polynomial we can get information about the
valuations of its roots.

\subsection{Power series}

Thus far we have worked with Newton polygons of polynomials, however we will eventually want to
know properties of roots of power series. Therefore, we would like to generalise all the work we
have done previously to power series.

The definition of Newton polygons of power series are formally identical, and can be constructed
via the same algoritheorem. However, because of the non-finiteness of degree we can run into some
interesting examples.
For example, consider the following power series, \cite[Page~260]{Gouvea}.
\[
f(x) = 1 + a x + ax^2 + \cdots +a x^d +\cdots \quad \text{for some } a \text{ with } \nu_p(a) > 0.
\]
Our set of interest is $\{(0,0),(n,\nu (a))  : \forall n \in \mathbb{N} \}.$ Doing the algoritheorem
gives a line that sweeps unbroken along the horizontal axis. Importantly, none of our points
$(i, \nu (a))$ lies on the line, and any increase in the slope will mean some points lie below the
line.
To solve this issue we adjust the algoritheorem slightly. As before, start with the vertical half-line
down from the origin. Then rotate this line counter-clockwise until one of the following happens:
\begin{enumerate}
    \item The line simultaneously hits infinitely many of the points we have plotted.
    In this case don't break the line and the polygon is complete.
    \item The line reaches a position where it can be rotated no further without leaving behind some
    points (as in example above). In this case this half lines forms the final segment of the
    polygon.
    \item The line hits a finite number of points. In this case, break at the last point that was
    hit and repeat the procedure.
\end{enumerate}

Defining slopes as before, we have a relation between the slope of the final segment and the
radius of convergence of the series, \cite[Lemma~7.4.8]{Gouvea}.
\begin{theorem} \label{RadofCon}
    Let $m$ be the supremum of the slopes appearing in the Newton polygon of a series
    $f(x) = 1 + a_1 x + a_2 x^2 + \cdots$. Then the radius of convergence of $f$ is $p^m.$
\end{theorem}

The generalisation of Theorem \ref{NPPoly} to power series comes as a corollary of the following,
\cite[proposition~7.4.10]{Gouvea}.

\begin{proposition}\label{NP}
    Let $f(x) = 1 + a_1 x + a_2 x^2 + \cdots$ be a power series. Let $m_1,m_2, \dots,m_k$ be the
    first $k$ slopes of the Newton polygon of $f(x)$, and assume that $f(x)$ converges on the
    closed ball of radius $c = p^{m_k}$ (from previous section, this means it is a restricted power
    series for parameter $c$ if $K$ is complete). Let $N$ be the $x$-coordinate of the right
    endpoint of the $k$-th segment of the Newton polygon. Then there exists a polynomial
    $g(x)$ of degree $N$ and a power series $h(x)$ such that
    \begin{enumerate}
        \item $f(x) = g(x) h(x)$,
        \item $\| f(x) - g(x) \|_c < 1$,
        \item $h(x)$ converges on the closed ball of radius $c$,
        \item $\| h(x) - 1\|_c <1$, and
        \item the Newton polygon of $g(x)$ is the same as the portion of the Newton polygon of
        $f(x)$ contained in the region $0 \leq x \leq N$.
    \end{enumerate}
\end{proposition}

And as promised here is the generalisation of Theorem~\ref{NPPoly}.

\begin{corollary}
    Let $f(x) = 1 + a_1 x + a_2 x^2 + \cdots$ be a power series which converges on the closed ball
    of radius $c = p^m$. Let $m_1,\dots, m_k$ be the slopes of the Newton polygon of $f(x)$ which
    are less than or equal to $m$, and let $i_1,\dots,i_k$ be their lengths. Then, for each $j$, the
    power series $f(x)$ has $i_j$ zeros with absolute value $p^{m_j}$, and there are no other zeros
    in the closed ball of radius $p^m.$
\end{corollary}

\subsection{Weierstrass preparation theorem}

This section will be dedicated to proving proposition~\ref{NP}.

The first important theorem we need to know is the $p$-adic Weierstrass preparation theorem,
\cite[Theorem~7.2.10]{Gouvea}.

\begin{theorem}\label{WP}
    Let $c$ be a positive real number and $f (x) = \sum a_n x^n$ be a power series with coefficients
    in $K$ such that $ \lvert a_n \rvert c^n \to 0$ as $n \to \infty$. Let $N$ be the number defined
    by the conditions
    \[
    \lvert a_N \rvert c^N = \max \lvert a_n \rvert c^n = \|f(x)\|_c \quad \text{and} \quad
    \lvert a_n \rvert c^n < \lvert a_N \rvert c^N \text{ for all } n > N.
    \]
    Then there exist a polynomial
    \[
    g(x) = b_0 + b_1x + \cdots + b_N x^N
    \]
    of degree $N$ and with coefficients in $K$, and a power series
    \[
    h(x) = 1 + d_1 x + d_2 x^2 + \cdots
    \]
    with coefficients in $K$, satisfying:
    \begin{itemize}
        \item $f(x) = g(x) h(x)$;
        \item $\lvert b_N \rvert c^N = \max \lvert b_n \rvert c^n$, that is
        $\| g (x) \|_c = \lvert b_N \rvert c^N$;
        \item $h(x) \in \mathcal{A}_c$, that is it is a restricted power series over $K$ for
        parameter $c$;
        \item $\lvert d_n \rvert c^n < 1$ for all $n \geq 1$, so that $\| h(x) - 1 \|_c < 1$; and
        \item $\| f(x) - g(x) \|_c < \| f(x)\|_c.$
    \end{itemize}
    In particular, $h(x)$ has no zeroes in $\overline{B}(0,c)$.
\end{theorem}

To prove this we will need to give a string of lemmas that will all come together.

We begin by showing that the Tate algebra with parameter $c$ is complete with respect to the Gauss
norm of parameter $c$, \cite[Lemma~7.2.7]{Gouvea}.

\begin{lemma}
    $\mathcal{A}_c$ is complete with respect to the Gauss norm $\|\cdot \|_c$.
\end{lemma}

\begin{proof}
    We need to show a Cauchy sequence in $\mathcal{A}_c$ converges with respect to the norm
    $\|\cdot\|_c$. Therefore, consider a sequence of power series
    \[
    f_i(x) = a_{i,0} + a_{i,1}x + a_{i,2}x^2 + \cdots.
    \]
    Saying this sequence is Cauchy means
    \[
    \forall \; \varepsilon>0, \; \exists \; M \text{ such that } \forall \; i,j > M
    \text{ we have } \|f_i (x) - f_j (x) \|_c < \varepsilon.
    \]
    Using the definition of the Gauss norm, this means that
    \[
    \max_{n} |a_{i,n}-a_{j,n} | c^n < \varepsilon \quad \forall \; i,j > M,
    \]
    and certainly,
    \[
    |a_{i,n} - a_{j,n}| < \varepsilon c^{-n}
    \]
    for each $n$ whenever $i,j > M.$ That is each of the sequences $(a_{i,n})_{i \in \N}$ forms a
    Cauchy sequence. Since $K$ is complete, this means they are convergent.
    Now consider
    \[
    g(x) = a_0 + a_1 x + a_2 x^2 + \cdots
    \]
    where $a_n = \lim_{i \to \infty} a_{i,n}$. We want to show this is the limit of the $f_n$ and
    is also in $\mathcal{A}_c$.

    For the first part, we know that if $i,j > M$ we have
    \[
    |a_{i,n} - a_{j,n} | c^n < \varepsilon
    \]
    and letting $j \to \infty$ it follows that if $i > M$
    \[
    | a_{i,n} - a_n| c^n \leq \varepsilon.
    \]
    Therefore, we have $\forall \; i > M$
    \[
    \| f_i (x) - g(x) \|_c = \max_n |a_{i,n} - a_n| c^n \leq \varepsilon
    \]
    and so $f_i(x) \to g(x)$ with respect to $\| \cdot \|_c.$

    Finally, since $f_ i \in \mathcal{A}_c$, there exists $N$ such that for all
    $n > N$, $|a_{i,n} | c^n \leq \varepsilon.$ Thus for all $n > N$
    \[
    | a_n |c^n = | a_n - a_{i,n} + a_{i,n}|c^n  \leq \max \{|a_n - a_{i,n}|c^n,|a_{i,n}|c^n\} \leq \varepsilon
    \]
    and therefore $|a_n|c^n \to 0$, which by definition means $g(x) \in \mathcal{A}_c$.
    With this we conclude the proof.
\end{proof}

Next we show that polynomials are dense in the Tate algebra, \cite[Lemma~7.2.8]{Gouvea}.
\begin{lemma}\label{PolyDense}
    The space of polynomials $K[x]$ is dense in $\mathcal{A}_c$.
\end{lemma}

\begin{proof}
    We need to show that any power series is the limit of a sequence of polynomials. Towards this
    let
    \[
    f(x) = a_0 + a_1x+a_2x^2 + \cdots
    \]
    be a power series in $\mathcal{A}_c$.
    Consider the sequence of truncations of $f(x)$,
    \begin{align*}
        f_0(x) &= a_0 \\
        f_1(x) &= a_0 + a_1 x \\
        f_2(x) &= a_0 + a_1 x + a_2 x^2 \\
        & \; \; \vdots \\
        f_k(x) &= a_0 + a_1 x + \cdots a_k x^k
    \end{align*}
    we want to show $f$ is the limit of this sequence. Now
    \[
    \| f(x) - f_k (x) \|_c = \max_{n>k} |a_n|c^n,
    \]
    which tends to zero as $k \to \infty$ since $\lim_{n \to \infty} |a_n|c^n = 0$. Thus
    $f_k (x) \to f(x)$ with respect to the Gauss norm and we conclude.
\end{proof}

Finally, we give a generalisation of \cite[Lemma~7.2.9]{Gouvea}, which is an analogue of the
Euclidean algoritheorem for Tate algebras.

\begin{lemma} \label{euclidalgoritheorem}
    Let $f(x) \in \mathcal{A}_c$, and let
    \[
    g(x) = b_0 + b_1 x + \cdots + b_N x^N
    \]
    be a polynomial with coefficients in $K$ satisfying
    \[
    \lvert b_N \rvert  c^N= \max_i (\lvert b_i \rvert c^i).
    \]
    Then there exist a power series $q(x) \in \mathcal{A}_c$ and a polynomial $r(x) \in K[x]$,
    of degree less that $N$, such that
    \[
    f(x) = g(x) q(x) + r(x)
    \]
    where $q(x)$ and $r(x)$ satisfy
    \[
    \|f(x) \|_c \geq \|g(x)\|_c \| q(x)\|_c \quad \text{and} \quad \|f(x) \|_c \geq \| r(x) \|.
    \]
    \end{lemma}

\begin{proof}
    Need to write up.
\end{proof}

With these theorems we are now in a position to prove the $p$-adic Weierstrass preparation theorem.

\begin{proof}[Proof of Theorem~\ref{WP}]
    The idea is to start with an approximate factorisation, and then improve it. If we prove by
    induction this is always possible, it will produce convergent sequences of polynomials and power
    series which in the limit gives the factorisation we want.

    \textbf{Base case:}
    To start the induction we need to find a $\delta < 1$ and an initial pair $g_1 (x)$ and
    $h_1 (x)$. Towards this consider
    \[
    g_1 (x) = \sum^{N}_{i = 0} a_i x^i \quad h_1(x) = 1.
    \]
    Then, since the assumption is that $\|f(x)\|_c = |a_N |c^N$ and the terms of higher degree are
    smaller, we have
    \[
    \| f(x) - \sum_{i=0}^N a_i x^i\|_c < \|f(x)\|_c
    \]
    and let $\delta$ be a measure of how much smaller:
    \[
    \|f(x) - \sum_{i=0}^N a_i x^i \|_c = \delta \|f(x) \|_c,
    \]
    clearly $0 < \delta < 1$. Then we can easily check that all our properties are satisfied.

    \textbf{Inductive step:}
    Let $\delta$ be a fixed real number, $0 < \delta < 1$. Suppose at stage $n$ we have a polynomial
    $g_n (x)$ and a power series $h_n (x)$ such that
    \begin{enumerate}
        \item $\|f(x) - g_n (x) h_n(x) \|_c \leq \delta^n \|f(x)\|_c$;
        \item $\deg g_n (x) = N$;
        \item $\|g_n(x) \|_c = |b_{n,N} | c^N$;
        \item $h_n (x) \in \mathcal{A}_c$;
        \item $\| h_n(x) - 1 \|_c \leq \delta^n$; and
        \item $\|f (x) - g_n (x) \|_c < \delta \|f(x)\|_c.$
    \end{enumerate}

    Firstly, $\|f(x) - g_n (x) \|_c \leq \delta \|f(x) \|$ implies $ \| f(x) \|_c = \| g(x)_n \|_c$,
    using the fact $\| \cdot \|_c$ is non-archimedean and that $0 < \delta < 1$.

    By Lemma \ref{euclidalgoritheorem} we can find a polynomial $r(x)$ of degree less than $N$ and a
    power series $q(x) \in \mathcal{A}_c$ such that
    \[
    f(x) - g_n (x) h_n (x) = q(x) g_n(x) + r(x)
    \]
    and
    \begin{align*}
    \| q(x) \|_c & \leq \frac{\| f(x) - g_n (x) h_n (x) \|_c}{\| g_n (x) \|_c} \leq
    \frac{\delta^n \|f_n(x)\|_c}{\|f(x) \|_c} = \delta^n \leq \delta\\
    \| r(x) \|_c &\leq \| f(x) - g_n (x) h_n (x) \|_c \leq
    \delta^n \| f(x) \|_c \leq \delta \| f(x) \|_c.
    \end{align*}
    Now consider
    \[
    g_{n+1} = g_n (x) + r(x) \quad \text{and} \quad h_{n+1}(x) = h_n (x) + q(x).
    \]
    We claim these are better approximations and satisfy the wanted properties.

    Firstly, since $\deg r(x) < N$ we must have $\deg g_{n+1} (x) = N = \deg g_n (x)$ and
    $g_{n+1}$ will also have leading term $b_{n,N}$.
    Moreover, since $\mathcal{A}_c$ forms a ring, $h_{n+1} \in \mathcal{A}_c$ (polynomials are
    trivially restricted power series).

    Next, we have
    \begin{align*}
        \| f(x) - g_{n+1}(x) \|_c &= \| f(x) - g_n (x) - r(x) \|_c \\
        & \leq \max \{ \|f(x) - g_n(x) \|_c, \| r(x) \|_c \} \\
        & \leq \delta \| f(x) \|_c
    \end{align*}
    which means by the same argument as for $g_n$
    \[
    \|g_{n+1} (x)\|_c = \| f(x) \|_n = \| g_n ( x) \|_c = |b_{n,N}| c^N = | b_{n+1,N} | c^N.
    \]
    We also have
    \begin{align*}
        \| h_{n+1} (x) - 1 \|_c & = \| h_n (x) - 1 + q(x) \|_c \\
        & \leq \max \{ \| h_n (x) - 1 \|_c , \| q(x) \|_c \} \\
        & \leq \delta .
    \end{align*}

    Now we want to show it indeed gives a better approximation:
    \begin{align*}
        f(x) &- g_{n+1} (x) h_{n+1}(x) = f(x) - (g_n (x) + r(x) ) (h_n(x) + q(x)) \\
        &= f(x) - g_n (x) h_n (x) - q(x) g_n(x) - r(x) h_n (x) - r(x) q(x) \\
        & = r(x) - r(x) h_n (x) - r(x) q(x) \\
        & = r(x) ( 1 - h_n (x) - q(x) )
    \end{align*}
    and therefore
    \begin{align*}
    \| f(x) - g_{n+1} (x) h_{n+1} (x) \|_c &= \| r(x) \|_c \|(1 - h_n(x)) - q(x) \|_c \\
    & \leq \delta \|f(x) \|_c \max\{ \|(1 - h_n (x)) \|_c, \|q(x)\|_c \} \\
    & \leq \delta^{n+1} \|f(x) \|_c.
    \end{align*}
    Since, $0 < \delta < 1$ this implies we do indeed have a better approximation.

    Therefore $g_{n+1}$ and $h_{n+1}$ satisfies all our properties with $\delta^n$ replaced by
    $\delta^{n+1}$.

    To conclude, we want to show these functions actually form the sequence we want.
    Firstly, they are Cauchy as
    \[
    \| q(x) \|_c \leq \delta^n \implies \|h_n (x) - h_{n+1}  (x) \|_c \leq \delta^n
    \]
    and
    \[
    \| r(x) \|_c \leq \delta^i \|f(x) \|_c \implies
    \|g_n (x) - g_{n+1} (x) \|_c \leq \delta^i \| f(x) \|.
    \]
    Therefore, as $\mathcal{A}_c$ is complete, we have they converge to a $g(x)$ and $h(x)$. It is
    then immediate via taking limits that these have the wanted properties.
\end{proof}

Finally, with this out the way we can prove proposition \ref{NP}.

\begin{proof}[Proof of proposition~\ref{NP}]
    We prove this by induction on $k$:\\
    If $k=1$, write the break point as $(N, m_1N)$. We know all points are on or above the line
    $y = m x$, and the ones where $j > N$ are strictly above it. This translates to
    \begin{itemize}
        \item $|a_j | (p^j)^{m_1} \leq 1$ for all $j$,
        \item $|a_N | (p^N)^{m_1} = 1$, and
        \item $|a_j | (p^{m_1})^j$ < 1 if $j > N.$
    \end{itemize}
    This says that $\|f(x) \|_{p^{m_1}} = 1$ and that the maximum is last realised at the degree
    $N$. Thus, we can apply the Weierstrass preparation theorem to conclude that there is a
    polynomial of degree $N$ and a power series $h(x)$ satisfying
    \[
    \| h(x) - 1 \|_{p^m_1} < 1 \quad \|f(x) - g(x)\|_{p^m_1} < \|f(x) \|_{p^m_1} \quad \text{and} \quad f(x) = g(x) h(x).
    \]
    Combining the middle equation with $\| f(x) \|_{p^{m_1}} < 1$ means
    $\| f(x) - g(x) \|_{p^{m_1}} < 1$.
    Now, using facts about Newton polygons of polynomials, it follows that the zeros of $f(x)$ in
    the closed ball of radius $p^m$ coincide with the those of $g(x)$. That is, we must have the
    Newton polygon of $g(x)$ coincides with the Newton polygon of $f(x)$ on $0 \leq x \leq N$.
    Finally, we have $f(x)$ converges on the closed ball of radius $p^{m_1}$, by Theorem
    \ref{RadofCon}, and thus so does $h(x).$
    \\
    By induction, assume we have $f(x) = g_{k-1}(x)h_{k-1}(x)$ for a polynomial $g_{k-1}(x)$ and
    power series $h_{k-1}(x)$, such that the Newton polygon of $g_{k-1}(x)$ coincides with the
    first $k-1$ segments of the polygon for $f(x)$ and $h_{k-1}(x)$ has no zeroes in the closed ball
    of radius $p^{m_{k-1}}$.
    \\
    Let us now consider the $k$-th segment. By definition, the $k$-th segment ending at $x = N$
    means that for any $i > N$, $(i, \nu_p (a_i))$ lies above the line of slope $m_k$ through the
    point $(N, \nu_p (a_N))$. This means that, writing $c = p^{m_k}$, $\| f(x) \|_{p^{c}} =
    \lvert a_N \rvert c^N$ and $\lvert a_n \rvert c^N > \lvert a_i \rvert c^i $ for any $i > N$.
    Thus, we can apply the $p$-adic Weierstrass preparation theorem to get a polynomial $g(x)$.
    It is then easy to see $g_1 (x) \mid g(x)$, and that its Newton polygon coincides with the
    relevant portion of the Newton polygon of $f(x).$
\end{proof}
