\section{Restricted Power Series}

Recall formal power series over a set $K$ in one variable, written $K \llbracket x \rrbracket $, are
infinite sums
\[
f(x) = \sum_{n=0}^\infty a_n x^n, \quad \forall i, \; a_i \in K.
\]
This can be generalized to multivariate formal power series as
\[
\sum a_{(n)} x^{(n)}
\]
where $x^{(n)} = x_1^{n_1} \cdots x_m^{n_m}$ with $n = \sum_{i = 1}^m n_i$, and the sum is taken
over all tuples $(n) = (n_1,\dots,n_m)$ of non-negative integers.

When $K$ is a ring, it is not hard to see that power series over $K$ form a ring, and we will often
refer to them as the ring of power series over $K$.

We want to introduce a convergence property on the terms $a_n x^n$, to do this we need a norm.

\begin{defn}
    We say a ring $R$ is a normed ring if we can equip it with a norm.
\end{defn}

We now restrict ourselves to normed rings $R$ and define a subset of power series with a convergence
property.

\begin{defn}
    Let $f$ be a power series over $R$ in one variable, write $f = \sum_{n=0}^\infty a_n x^n$, and
    let $c$ be a real number. We say $f$ is a restricted power series of parameter $c$ over $K$ if
    \[
    \lim_{n \to \infty} \lvert a_n \rvert c^n = 0.
    \]
\end{defn}

It would be nice if restricted power series of weight $c$ were a ring, that is if we added, or
multiplied, two restricted power series the resulting power series would also be restricted of
weight $c$.

If the norm of $R$ has the non-archimedean property, it indeed turns out this is true.
The non-archimedean property is key here; restricted power series form an additive group regardless
of the norm on $K$, but a stronger bound than the triangle inequality is necessary for the
multiplication to be closed. This can be seen by the following example:

\begin{example}

Let $f = \sum_{n=0}^\infty \frac{(-1)^n}{\sqrt{n+1}}x^n$ be a power series over $\mathbb{R}$
equipped with the standard norm. Then it is trivial to see that $f$ is a restricted power series for
the parameter 1. However, the product $f^2$ has coefficients
\[
b_n = (-1)^n \sum_{k=0}^n \frac{1}{(n - k -1)(k+1)},
\]
and it can be shown that
\[
\lvert b_n\rvert \geq \sum_{k = 0}^n \frac{2}{n + 2} = \frac{2 (n + 1)}{(n + 2)}.
\]
Therefore $\lim_{n \to \infty} \lvert b_n \rvert \geq \lim_{n \to \infty} \frac{2(n+1)}{(n+2)}=2$,
that is $f^2$ is not a restricted power series for parameter 1, and we see multiplication is not
closed.

\end{example}

\begin{theorem}
  Restricted power series over a normed ring $R$ form an additive group.
\end{theorem}

\begin{theorem}
  Restricted power series over a non-archimedean normed ring $R$ form a ring.
\end{theorem}

We now suppose that $K$ is a non-archimedean complete field; that is a field equipped with a metric
that is complete and whose metric has the non-archimedean property as defined earlier, like $\Q_p$.

\begin{defn}
    When $K$ is a non-archimedean complete field we denote the ring of restricted power series with
    parameter $c$ over $K$ by $\mathcal{A}_c(K)$ and call it the Tate algebra over $K$ of parameter
    $c$.
\end{defn}
