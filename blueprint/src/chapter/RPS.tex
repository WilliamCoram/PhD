\chapter{Restricted Power Series}

\section{Definition}

Recall formal multivariate power series over a set $K$ in $n$ variables, written
$K [ \! [ x_1, \dots, x_n ] \! ]$ are infinite sums
\[
  f(x) = \sum a_{(i_1, \cdots, i_n)} x_1^{i_1} \cdots x_n^{i_n},
\]
where the sum is taken over all $n$-tuples $(i_1, \cdots i_n)$ with $i_j \in \mathbb{N}$ and
$a_{(i_1, \cdots, i_n)} \in K$.

When $K$ is a ring, formal multivariate power series in $n$ variables form a ring, and we will refer
to them as the ring of power series over $K$ in $n$ variables. We again denote such
rings using $K [ \! [ x_1, \dots, x_n ] \! ]$.

We define a subset of $K [ \! [ x_1, \dots, x_n ] \! ]$ that consists of power series whose
coefficeients approach zero in the cofinite filter.

\begin{definition}
  \label{IsRestricted}
  \lean{IsRestricted}
  \leanok
  Let $f$ be a power series over a normed ring $R$ in one variable, write
  $f(x) = \sum a_{(i_1, \cdots, i_n)} x_1^{i_1} \cdots x_n^{i_n},$, and let
  $c = (c_1, \cdots, c_n)$ be an $n$-tuple of real numbers.
  We say $f$ is a restricted power series of parameter $c$ over $R$ in $n$ variables if
  \[
    \lim_{n \to \infty} \lvert a_{(i_1, \cdots, i_n)} \rvert \prod_{j=1}^{n} c_j^{i_j} = 0
  \]
  in the cofinite topology.

  We denote the set of restricted power series of parameter $c$ over $R$ in $n$ variables by
  $R_c [ \! [ x_1, \dots, x_n ] \! ]$.
\end{definition}

\begin{remark}
  Note
  \[
    \lim \lvert a_{(i_1, \cdots, i_n)} \rvert \prod_{j=1}^{n} c_j^{i_j} = 0
  \]
  in the cofinite topology, if for all $0 < \epsilon$ there are finitely many $n$-tuples that have
  $\left \lvert \lvert a_{(i_1, \cdots, i_n)} \rvert \prod_{j=1}^{n} c_j^{i_j} \right \rvert$
  greater than $\varepsilon$.
\end{remark}

\section{Additive Group}

It turns out that restricted power series of parameter $c$ over a normed ring $R$ in $n$ variables
are not just a set, in fact they form an additive group.

\begin{theorem}
  For any normed ring $R$, $n \in \mathbb{N}$ and $c \in \mathbb{R}^n$. Then
  $R_c [ \! [ x_1, \dots, x_n ] \! ]$ forms an additive group.
\end{theorem}

Let $R$ be a normed ring, $n \in \N$ and $c \in \mathbb{R}^n$
We prove this in lean using the following lemmas.

\begin{lemma}
  \label{isRestricted_iff_abs}
  \lean{isRestricted_iff_abs}
  \leanok
  \uses{IsRestricted} %% Do I need to say uses in all the statements?
  $f \in R_c [ \! [ x_1, \dots, x_n ] \! ]$ if and only if $f \in R_{\lvert c \rvert}
  [ \! [ x_1, \dots, x_n ] \! ]$.
  That is we can always consider $c$ to have positive entries.
\end{lemma}

\begin{proof}
  \[
    \left \lvert \lvert a_{(i_1, \cdots, i_n)} \rvert \prod_{j=1}^{j=n} c_j^{i_j} \right \rvert =
    \left \lvert \lvert a_{(i_1, \cdots, i_n)} \rvert \prod_{j=1}^{j=n} \lvert c \rvert _j^{i_j}
    \right \rvert.
  \]
\end{proof}

\begin{lemma}
  \label{zero}
  \lean{zero}
  \leanok
  $0 \in R_c [ \! [ x_1, \dots, x_n ] \! ]$.
\end{lemma}

\begin{proof}
  Substitute the zero power series into the definition - every coefficient is 0, thus we are
  immediately done.
\end{proof}

\begin{lemma}
  \label{monomial}
  \lean{monomial}
  \leanok
  Any monomial $m = a_{(i_1, \cdots, i_n)} x_1^{i_1} \cdots x_n^{i_n}$ is restricted.
\end{lemma}

\begin{proof}
  Only one coefficient is non-zero.
\end{proof}

\begin{lemma}
  \label{One}
  \lean{One}
  \leanok
  \uses{monomial}
  $1 \in R_c [ \! [ x_1, \dots, x_n ] \! ]$.
\end{lemma}

\begin{proof}
  1 is a monomial.
\end{proof}

\begin{lemma}
  \label{add}
  \lean{add}
  \leanok
  \uses{IsRestricted, isRestricted_iff_abs}
  If $f, g \in R_c [ \! [ x_1, \dots, x_n ] \! ]$ then $f+g \in R_c [ \! [ x_1, \dots, x_n ] \! ]$.
\end{lemma}

\begin{proof}
  Squeeze theorem.
\end{proof}

\begin{lemma}
  \label{neg}
  \lean{neg}
  \leanok
  \uses{IsRestricted, isRestricted_iff_abs}

\end{lemma}

\begin{proof}
  \[
    \left \lvert \lvert a_{(i_1, \cdots, i_n)} \rvert \prod_{j=1}^{n} c_j^{i_j} \right \rvert =
    \left \lvert \lvert - a_{(i_1, \cdots, i_n)} \rvert \prod_{j=1}^{n} c_j^{i_j} \right \rvert.
  \]
\end{proof}

\iffalse

Formal power series over a set $K$ in one variable, written $K [ \! [ x ] \! ] $, are
infinite sums
\[
f(x) = \sum_{n=0}^\infty a_n x^n, \quad \forall i, \; a_i \in K.
\]
When $K$ is a ring, we can show power series over $K$ form a ring, and we will often refer to them
as the ring of power series over $K$, denoted by $\mathcal{F}_K$.

We define a subset of $\mathcal{F}$ that consists of power series whose coefficeients approach zero,
to do this we first better make sure $K$ is a normed ring

\begin{definition*}
A normed ring is a pair $(R,\lvert \cdot \rvert)$, of a ring and a norm.
\end{definition*}

With this we define,

\begin{definition}
    \label{PowerSeries.IsRestricted}
    \lean{PowerSeries.IsRestricted}
    \leanok
    Let $f$ be a power series over a normed ring $R$ in one variable, write
    $f = \sum_{n=0}^\infty a_n x^n$, and let $c$ be a real number.
    We say $f$ is a restricted power series of parameter $c$ over $K$ if
    \[
    \lim_{n \to \infty} \lvert a_n \rvert c^n = 0
    \]
    in the cofinite topology.
\end{definition}

Note this is not the same as convergent power series in the usual sense unless $R$ is complete,
since in complete spaces Cauchy sequences converge.

It turns out that restricted power series of parameter $c$ are not just a set, in fact we have the
following.

\begin{theorem}
  \label{PowerSeries.Restricted.IsGroup}
  \lean{PowerSeries.Restricted.IsGroup}
  \leanok
  Restricted power series over a normed ring $R$ form an additive group.
\end{theorem}

\section{Additive Group}

To prove that they form a ring in

Moreover, if the norm of $R$ has the non-archimedean property,

\begin{definition*}
  We say a norm is non-archimedean if
  \[
  \lvert a + b \rvert \leq \max \{ \lvert a \rvert, \lvert b \rvert\}.
  \]
  Else the norm is archimedean.
\end{definition*}

we can strengthen this lemma.

\begin{theorem}
  \label{PowerSeries.Restricted.IsRing}
  \lean{PowerSeries.Restricted.IsRing}
  \leanok
  Restricted power series over a non-archimedean normed ring $R$ form a ring.
\end{theorem}

The importance of the non-archimedean property can be seen in the following example.

Let $f = \sum_{n=0}^\infty \frac{(-1)^n}{\sqrt{n+1}}x^n$ be a power series over $\mathbb{R}$
equipped with the standard norm. Then it is immediate to see that $f$ is a restricted power series
for the parameter 1. However, the product $f^2$ has coefficients
\[
b_n = (-1)^n \sum_{k=0}^n \frac{1}{(n - k -1)(k+1)},
\]
and it can be shown that
\[
\lvert b_n\rvert \geq \sum_{k = 0}^n \frac{2}{n + 2} = \frac{2 (n + 1)}{(n + 2)}.
\]
Therefore $\lim_{n \to \infty} \lvert b_n \rvert \geq \lim_{n \to \infty} \frac{2(n+1)}{(n+2)}=2$,
that is $f^2$ is not a restricted power series for parameter 1, and we see multiplication is not
closed.

We now suppose that $K$ is a non-archimedean complete field; that is a field equipped with a metric
that is complete and whose metric has the non-archimedean property as defined earlier. We give a
new name to restricted power series with these $K$.

\begin{definition}
    When $K$ is a non-archimedean complete field we denote the ring of restricted power series with
    parameter $c$ over $K$ by $\mathcal{A}_c(K)$ and call it the Tate algebra over $K$ of parameter
    $c$.
\end{definition}

As mentioned earlier, this set is equivalent to power series that converge on the ball of radius
$c$, and so sometimes in literature the Tate algebra may be referenced as convergent power series.

\fi
