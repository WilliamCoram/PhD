\chapter{Quaternionic Modular Forms}

\section{Definition}

We define quaternionic modular forms over definite quaternion algebras explicitely as the following
space.

\begin{definition}
  Let $\kappa : \mathbb{Z}_p^\times \to \mathcal{O}_p^\times$ be a locally analytic character.
  Given $\alpha \in \mathbb{N}$, let
  \[
    \Sigma_\alpha = \left\{ \gamma =  \begin{pmatrix}
      a & b \\
      c & d
    \end{pmatrix}
    \in \text{M}_2(\mathbb{Z}_p) : p^\alpha \mid c, \; p \nmid d, \; \det (\gamma) \neq 0 \right\}.
  \]
  The weight $\kappa$ action of $\gamma =  \begin{pmatrix} a & b \\ c & d \end{pmatrix} \in
  \Sigma_\alpha$ on the Tate algebra $\mathcal{A}_p$ with coefficients in $\mathbb{C}_p$ is given by
  the continuous $\mathbb{C}_p$-linear extension of the map sending
  \[
    z^k \mapsto \frac{\kappa(c z + d)}{(cz + d)^2} \left( \frac{az+b}{cz+d} \right)^k
  \]
  and given $f(z) \in \mathcal{A}_p$ we write $(f||_\kappa \gamma)(z)$ for this action.

  Fix $\alpha$ and $\kappa$ in the above, let $D$ be a definite quaternion algebra and let $U$ be
  an open compact subgroup of $D_f^\times$, where $D_f = D  \otimes_{\mathbb{Q}} \mathbb{A}_f$ is
  $D$ over the finite adeles, of wild level $\geq p^\alpha$; that is the projection
  $U \to D_{p}^\times$ is contained in $\Sigma_\alpha.$
  Now let $A$ be any right $\Sigma_\alpha$-module.

  Then the level $U$, weight $\kappa$ space of automorphic forms is the space
  \[
    \mathcal{L}(U,A) = \{ \varphi : D_f^\times \to A  \;  | \; \varphi (dgu) = \varphi (g)
    ||_\kappa u_p, \; \forall d \in D^\times, \; g \in D^\times_f, \; u \in U \}.
  \]
\end{definition}

\section{Finite Dimensional}

\begin{lemma}
  Using the above notation and supposing $D_f^\times = \coprod_{i \in I} D^\times c_i U$
  is the decomposition into finite double cosets, then
  \[
    \mathcal{L}(U,A) \cong \bigoplus_{i \in I} A^{\Gamma_i}
  \]
  as a $\mathbb{C}_p$ vector space, where $\Gamma_i = c_i^{-1} D^\times c_i \cap U$.
\end{lemma}

\section{Hecke Operators}

We can define Hecke operators on these automorphic forms. We will keep the notation from the above
definitions.
Given $\varphi \in \mathcal{L}(U,A)$ we define a right action of $U$ on $\mathcal{L}(U,A)$:
\[
  (\varphi |_\kappa u) (g) := \varphi (gu^{-1}) ||_\kappa u_p.
\]

With this we have:
\begin{definition}
  Let $\nu \in D^\times_f$ such that $\nu_p \in \Sigma_\alpha$. Then the double coset $U\nu U$ may
  be written as a disjoint union
  \[
    U \nu U = \coprod_{t \in T} U \nu_t
  \]
  where $T$ is a finite set, and $\nu_t \in D_f^\times$. The Hecke operator is the map
  $[U \nu U] : \mathcal{L}(U,A) \to \mathcal{L}(U,A)$ given by
  \[
    [U \nu U] \varphi := \sum_{t \in T} \varphi |_\kappa \nu_t.
  \]
\end{definition}

\begin{definition}
  We set $\varpi_l$ to be the element of $\mathbb{A}_f$ which is $l$ at  $l$ and 1 at all other
  places, and set $\eta_l = \begin{pmatrix} \varpi_l & 0 \\ 0 &1 \end{pmatrix}$.\\
  The standard Hecke operators are then given by
  \[
    T_l := [U \eta_l U] \quad \text{and} \quad S_l:=[U \varpi U]
  \]
  and if $l = p$, we write $T_p = U_p$.
\end{definition}
